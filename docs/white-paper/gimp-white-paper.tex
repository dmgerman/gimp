\documentclass[12pt]{article}

\usepackage{times}

\let\slash=/
\catcode`\/=\active
\def/{\futurelet\next\slashswitch}
\def\slashswitch{\ifx\next/\let\next=\udblslash\else\let\next=\uslash\fi
\next}
\def\uslash{\slash\penalty\exhyphenpenalty}
\def\udblslash/{\slash\slash\penalty\exhyphenpenalty}
\def\url{\begingroup\tt}
\def\endurl{\endgroup}

\def\gimp{{\sc gimp}}
\def\gnu{{\sc gnu}}
\def\gpl{{\sc gpl}}

\newcommand{\FIXME}[1]{\emph{*** FIXME: #1 ***}}

\begin{document}

\title{GNU Image Manipulation Program\\
	Technology White Paper}
\date{}
\maketitle

\part{Introduction}

The {\bf G}NU\footnote{Please see \url http://www.gnu.org \endurl for
information 
on the \gnu\ Project.} {\bf I}mage {\bf M}anipulation {\bf P}rogram
(\gimp) is a powerful tool for the preparation and manipulation of
digital images.  The \gimp\ provides the user with a wide variety of
image manipulation, painting, processing, and rendering tools.  The
key to the \gimp's power lies in its flexible core and easily
extensible design.  The \gimp's open design and extensible
architecture make for a very powerful end product which can continue
to be extended to meet the needs of the photo compositor, image
retoucher, web graphics designer, or digital illustrator for a very
long time.

The \gimp's extensible plug-in architecture allows for image
manipulation procedures and other functionality to be easily added
without requiring any change to the application core.  A plug-in can
provide functionality as simple as rotating an image, or as
complicated as rendering iterated function system fractals.  There are
nearly 150 plug-ins available in version 1.0, and more are sure to
follow.

The plug-in architecture also allows the \gimp\ to support a wide
variety of file formats.  File operations are implemented by special
``file plug-ins'', allowing additional file formats to be added
without modification to the core.  File formats supported in version
1.0 include the popular GIF and JPEG standards, as well as PNG, TIFF,
XPM, SGI, PCX, and Windows BMP.

An innovative tile management system allows the \gimp\ to be used to
edit images much larger than can be stored in system memory; the
user's available disk space is the only real limit to the size of the
image a user can edit.

While the \gimp's primary emphasis is on image manipulation, it also
offers a complete set of painting tools for use in image creation.
Version 1.0 offers pencil, paintbrush, eraser, airbrush, and cloning
tools, as well as a variable-strength convolver.  All of these tools
(except the eraser) can be operated in any of the \gimp's 15 painting
modes.  A powerful gradient generator, with a very versatile custom
gradient editor, makes colored blends easy.

A full battery of image manipulators are available, including
rotation, scaling, translation, color, brightness, and contrast
adjustment, and gamma correction.  In addition, a great many other
useful transformations are available as plug-ins, and the ease of
extensibility here means that new capabilities are being added all the
time.

Finally, the most impressive feature of the \gimp\ is that it is
available under the terms of the GNU General Public License
(\gpl).\footnote{Please see \url http://www.gnu.org/copyleft/gpl.html
\endurl for the full text of
the \gpl.}  The entire source code is freely available and
distributed.  This openness has fostered a very active development
community and large user base, out of which a superior product has
arisen.

\part{Core architecture}

\section{Images}

The basic operating element of any digital image editor is the {\it
image}.  In the \gimp, images are constructed out of {\it layers},
which are stacked on top of on another through a process called {\it
composition} to produce a {\it projection}, which is what is displayed
to the user.  In addition to having any number of layers, a \gimp\
image may have one or more user-defined {\it channels}, as well as a
{\it selection mask}.  Channels and selection masks are discussed more
later.  Together, these three things (layers, channels, and selection
masks) along with {\it layer masks} (also to be discussed later) are
known as {\it drawables}, because the drawing tools work on all of
them.\footnote{There is presently no way to draw directly onto a
selection mask using the user interface.}

Images in the \gimp\ are typed, and there are presently three types of
image: RGB, grayscale, and indexed.  The type of image determines the
representation of the pixels in the image.  In an RGB image, each
pixel of the composited image is represented by a 24-bit RGB tuple;
all 16 million possible colors are potentially available in the
composited image.  Grayscale images are monochromatic, and each pixel
is a single 8-bit gray value, yielding 256 shades of gray.  Indexed
images represent each pixel as an index into a color table, each entry
of which is a 24-bit RGB tuple.  The type of all layers within an
image must be compatible with the image type.

\section{Drawables}

A {\it drawable} is a planar array of pixel data; however, the
contents of a drawable need not necessarily be used solely for
rendering as pixel data.  Each drawable contains from one to four
data channels (not to be confused with the channels spoken of
elsewhere in this document), depending on the type of the drawable.
Each data channel is one byte deep.

There are six types of drawables (see table \ref{tab:drawables}).

\begin{table}
\centering
\begin{tabular}{lcl}
Type & Data channels & Contents of channels \\
\hline
RGB & 3 & red, green, blue \\
RGB w/ alpha & 4 & red, green, blue, alpha \\
Grayscale & 1 & intensity \\ 
Grayscale w/ alpha & 2 & intensity, alpha \\
Indexed & 1 & color index \\
Indexed w/ alpha & 2 & color index, alpha \\
\end{tabular}
\caption{Drawable Types}
\label{tab:drawables}
\end{table}

\subsection{Layers}

Each {\it layer} is a drawable.  Layers of any type are possible, but
the type of a layer must be compatible with the type of the image of
which it is a part.  A layer type is compatible with an image type if
the two are the same, or if the layer type is the same as the image
type with an added alpha channel.  Every layer is part of exactly one
image.

\subsubsection{Layer masks}

\subsection{Channels}

The term ``channels'' actually refers to three different things in the
\gimp: layer masks, selection masks, and ``generic'' channels.  In all
three cases, a channel consists of a rectangular array of byte values;
the interpretation of these values varies depending on the type of
channel.

In addition to these channel types, each image also has either
one or three ``virtual'' channels (three for RGB images, one for
grayscale and indexed).  This/these are merely the pixel array
resulting from the layer composition process, and is made available to
support the ``sample merged'' functionality in some of the tools.
Also, with RGB images, it is possible to deselect one or more of the
red, green or blue channels; doing so will cause that portion of the
colorspace to be excluded from the composition process.

\section{Procedures}

The \gimp\ core consists of 215\footnote{In version 0.99.16.  This
number tends to go up with time.} procedures which operate on images
in a great variety of ways.
\FIXME{General overview}

\section{Tools}

\FIXME{overview, architecture}

\subsection{Gradient tool}

The gradient tool lets the user easily create color gradients.  It
supports several gradient types: linear, bilinear, radial, square,
conical, and the special ``shapeburst'' mode, which makes the gradient
follow the shape of the active selection.

Normally, the blend tool creates color gradients between the current
foreground and background colors.  It supports three blending modes:
linear interpolation between the RGB components, linear interpolation
between the HSV versions of the colors (for rainbow effects), and
blending from the foreground color to transparent.

Gradients are calculated with respect to a direction vector.  The user
can specify whether the gradient is to be rendered normally, or
repeated along the direction vector using a sawtooth or triangular
wave pattern.

A unique feature of the \gimp's gradient tool is its support for
user-defined, custom color gradients.  The \gimp\ sports a
fully-featured color gradient editor that lets the user create color
gradients with an arbitrary number of color transitions.

A custom gradient is represented internally as a list of contiguous,
non overlapping segments that define a partition of the range $[0, 1]$.  
Each segment has the following properties:

\begin{itemize}

\item Left and right colors that blend smoothly inside the segment.

\item An off-center midpoint, which can be used to bias the color
blend to the left or to the right.

\item A blending function, which can be linear, curved (with an
exponent), sinuosidal, or spherical (increasing and decreasing).

\item A coloring type, which can be RGB interpolation, clockwise or
couterclockwise HSV interpolation.

\end{itemize}

The user can drag the segments' endpoints left and right.  Segments
can be inserted and deleted at any time.  The gradient editor provides
several useful functions for manipulating segments (split, flip,
replicate) that make creation of custom gradients easy and convenient.
The color segments support full transparency information, making for
even more flexible gradients.

To avoid sampling artifacts (the ``jaggies''), the gradient rendering
engine supports adaptive supersampling with customizable threshold and
recursion depth parameters.  With adaptive supersampling even the most
complex custom gradients will be rendered smoothly without
artifacting.

\FIXME{drawing or screenshot?}



\end{document}
