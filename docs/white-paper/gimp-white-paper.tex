\documentclass[12pt]{article}

\usepackage{times}

\let\slash=/
\catcode`\/=\active
\def/{\futurelet\next\slashswitch}
\def\slashswitch{\ifx\next/\let\next=\udblslash\else\let\next=\uslash\fi
\next}
\def\uslash{\slash\penalty\exhyphenpenalty}
\def\udblslash/{\slash\slash\penalty\exhyphenpenalty}
\def\url{\begingroup\tt}
\def\endurl{\endgroup}

\def\gimp{{\sc gimp}}
\def\gnu{{\sc gnu}}
\def\gpl{{\sc gpl}}

\newcommand{\FIXME}[1]{\emph{*** FIXME: #1 ***}}

\begin{document}

\title{GNU Image Manipulation Program\\
	Technology White Paper}
\date{}
\maketitle

\part{Introduction}

The {\bf G}NU\footnote{Please see \url http://www.gnu.org \endurl for
information 
on the \gnu\ Project.} {\bf I}mage {\bf M}anipulation {\bf P}rogram
(\gimp) is a powerful tool for the preparation and manipulation of
digital images.  The \gimp\ provides the user with a wide variety of
image manipulation, painting, processing, and rendering tools.  The
key to the \gimp's power lies in its flexible core and easily
extensible design.  The \gimp's open design and extensible
architecture make for a very powerful end product which can continue
to be extended to meet the needs of the photo compositor, image
retoucher, web graphics designer, or digital illustrator for a very
long time.

The \gimp's extensible plug-in architecture allows for image
manipulation procedures and other functionality to be easily added
without requiring any change to the application core.  A plug-in can
provide functionality as simple as rotating an image, or as
complicated as rendering iterated function system fractals.  There are
nearly 150 plug-ins available in version 1.0, and more are sure to
follow.

The plug-in architecture also allows the \gimp\ to support a wide
variety of file formats.  File operations are implemented by special
``file plug-ins'', allowing additional file formats to be added
without modification to the core.  File formats supported in version
1.0 include the popular GIF and JPEG standards, as well as PNG, TIFF,
XPM, SGI, PCX, and Windows BMP.

An innovative tile management system allows the \gimp\ to be used to
edit images much larger than can be stored in system memory; the
user's available disk space is the only real limit to the size of the
image a user can edit.

While the \gimp's primary emphasis is on image manipulation, it also
offers a complete set of painting tools for use in image creation.
Version 1.0 offers pencil, paintbrush, eraser, airbrush, and cloning
tools, as well as a variable-strength convolver.  All of these tools
(except the eraser) can be operated in any of the \gimp's 15 painting
modes.  A powerful gradient generator, with a very versatile custom
gradient editor, makes colored blends easy.

A full battery of image manipulators are available, including
rotation, scaling, translation, color, brightness, and contrast
adjustment, and gamma correction.  In addition, a great many other
useful transformations are available as plug-ins, and the ease of
extensibility here means that new capabilities are being added all the
time.

Finally, the most impressive feature of the \gimp\ is that it is
available under the terms of the GNU General Public License
(\gpl).\footnote{Please see \url http://www.gnu.org/copyleft/gpl.html
\endurl for the full text of
the \gpl.}  The entire source code is freely available and
distributed.  This openness has fostered a very active development
community and large user base, out of which a superior product has
arisen.

\part{Core architecture}

\FIXME{General overview}

\section{Tools}

\FIXME{overview, architecture}

\subsection{Gradient tool}

The gradient tool lets the user easily create color gradients.  It
supports several gradient types: linear, bilinear, radial, square,
conical, and the special ``shapeburst'' mode, which makes the gradient
follow the shape of the active selection.

Normally, the blend tool creates color gradients between the current
foreground and background colors.  It supports three blending modes:
linear interpolation between the RGB components, linear interpolation
between the HSV versions of the colors (for rainbow effects), and
blending from the foreground color to transparent.

Gradients are calculated with respect to a direction vector.  The user
can specify whether the gradient is to be rendered normally, or
repeated along the direction vector using a sawtooth or triangular
wave pattern.

A unique feature of the \gimp's gradient tool is its support for
user-defined, custom color gradients.  The \gimp\ sports a
fully-featured color gradient editor that lets the user create color
gradients with an arbitrary number of color transitions.

A custom gradient is represented internally as a list of contiguous,
non overlapping segments that define a partition of the range $[0, 1]$.  
Each segment has the following properties:

\begin{itemize}

\item Left and right colors that blend smoothly inside the segment.

\item An off-center midpoint, which can be used to bias the color
blend to the left or to the right.

\item A blending function, which can be linear, curved (with an
exponent), sinuosidal, or spherical (increasing and decreasing).

\item A coloring type, which can be RGB interpolation, clockwise or
couterclockwise HSV interpolation.

\end{itemize}

The user can drag the segments' endpoints left and right.  Segments
can be inserted and deleted at any time.  The gradient editor provides
several useful functions for manipulating segments (split, flip,
replicate) that make creation of custom gradients easy and convenient.
The color segments support full transparency information, making for
even more flexible gradients.

The gradient rendering engine supports adaptive supersampling with
customizable threshold and recursion depth parameters.  Using adaptive
supersampling means that even the most complex custom gradients will
be rendered without artifacts or ``jaggies.''

\FIXME{drawing or screenshot?}



\end{document}
